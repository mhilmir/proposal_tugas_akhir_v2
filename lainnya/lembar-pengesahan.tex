\chapter*{LEMBAR PENGESAHAN}

% Menyembunyikan nomor halaman
\thispagestyle{empty}

\begin{center}
  % Ubah kalimat berikut dengan judul tugas akhir
  \textbf{KALKULASI ENERGI PADA ROKET LUAR ANGKASA BERBASIS \emph{ANTI-GRAVITASI}}
\end{center}

\begingroup
% Pemilihan font ukuran small
\small

\begin{center}
  % Ubah kalimat berikut dengan pernyataan untuk lembar pengesahan
  \textbf{PROPOSAL TUGAS AKHIR} \\
  Diajukan untuk memenuhi salah satu syarat memperoleh gelar
  Sarjana Teknik pada
  Program Studi S-1 Teknik Komputer \\
  Departemen Teknik Komputer \\
  Fakultas Teknik Elektro dan Informatika Cerdas \\
  Institut Teknologi Sepuluh Nopember
\end{center}

\begin{center}
  % Ubah kalimat berikut dengan nama dan NRP mahasiswa
  Oleh: \textbf{Mochammad Hilmi Rusydiansyah} \\
  NRP. 5024211008
\end{center}

\begin{center}
  Disetujui Oleh:
\end{center}

\vspace{10ex}

\begingroup
% Menghilangkan padding
\setlength{\tabcolsep}{0pt}

\noindent
\begin{tabularx}{\textwidth}{X c}
  % Ubah kalimat-kalimat berikut dengan nama dan NIP dosen pembimbing pertama
  Muhtadin, S.T., M.T.      &                 \\
  NIP: 19810609200912 1 003    & (Pembimbing)    \\
                                &                 \\
                                &                 \\
                                &                 \\
  % Ubah kalimat-kalimat berikut dengan nama dan NIP dosen pembimbing kedua
  Wernher von Braun, S.T., M.T. &                 \\
  NIP: 19230323 197706 1 001    & (Ko-Pembimbing) \\
\end{tabularx}
\endgroup

\vspace{\fill}

\begin{center}
  Mengetahui,\\
  % Ubah kalimat berikut dengan nama departemen
  Kepala Departemen Teknik Komputer FTEIC-ITS\\
  \vspace{10ex}
  % Ubah kalimat berikut dengan jabatan kepala departemen
  \underline{Dr. Arief Kurniawan, S.T., M.T. }\\
  NIP 19740907200212 1 001\\
  \vspace{10ex}
  % Ubah text dibawah menjadi tempat dan tanggal
  \textbf{SURABAYA} \\
  \textbf{Februari, 2025}
\end{center}
\endgroup
