\chapter*{ABSTRAK}
\begin{center}
  \large
  \textbf{\emph{AUTONOMOUS GRASPING} PADA \emph{QUADRUPED} ROBOT DENGAN INTERAKSI BERBASIS \emph{TASK LEVEL}}
\end{center}
\addcontentsline{toc}{chapter}{ABSTRAK}
% Menyembunyikan nomor halaman
\thispagestyle{empty}

\begin{flushleft}
  \setlength{\tabcolsep}{0pt}
  \bfseries
  \begin{tabular}{ll@{\hspace{6pt}}l}
  Nama Mahasiswa / NRP&:& Mochammad Hilmi Rusydiansyah / 5024211008\\
  Departemen&:& Teknik Komputer FTEIC - ITS\\
  Dosen Pembimbing&:& 1. Muhtadin, S.T., M.T.\\
  & & 2.  Prof. Dr. Ir. Mauridhi Hery Purnomo, M.Eng.\\
  \end{tabular}
  \vspace{4ex}
\end{flushleft}
\textbf{Abstrak}

% Isi Abstrak
Robot \emph{quadruped} semakin banyak digunakan dalam berbagai aplikasi karena
memiliki mobilitas tinggi dan mampu beroperasi di medan yang beragam.
Namun, sebagian besar robot \emph{quadruped} yang tersedia saat ini hanya difokuskan pada mobilitas
dan persepsi lingkungan tanpa dilengkapi dengan aktuator untuk manipulasi objek.
Jika robot \emph{quadruped} dilengkapi dengan lengan robot dan \emph{gripper},
pengendalian secara manual menjadi tantangan tersendiri,
terutama dalam skenario jarak jauh yang memerlukan kendali yang kompleks.
Penelitian ini bertujuan untuk mengembangkan sistem \emph{autonomous grasping}
pada robot \emph{quadruped} dengan interaksi berbasis \emph{task level}.
Sistem yang diusulkan mencakup integrasi perangkat keras berupa lengan robot dan \emph{gripper},
serta pengembangan kontrol bertingkat yang mencakup tahap mendekati objek,
mendeteksi objek, memilih objek, hingga melakukan \emph{grasping} secara otonom.
Algoritma terkini seperti GraspNet akan diterapkan untuk memungkinkan robot
melakukan \emph{grasping} secara mandiri tanpa campur tangan langsung dari operator.
Dengan sistem ini, robot \emph{quadruped} diharapkan mampu berinteraksi dengan lingkungannya secara lebih efektif,
membuka peluang penggunaan yang lebih luas dalam berbagai bidang
seperti pencarian dan penyelamatan, industri, dan eksplorasi lingkungan yang kompleks.

\vspace{2ex}
\noindent
\textbf{Kata Kunci: \emph{Autonomous Grasping, Robot Quadruped, Task-Level Interaction}}