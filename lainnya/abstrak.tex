\chapter*{ABSTRAK}
\begin{center}
  \large
  \textbf{\emph{AUTONOMOUS GRASPING} PADA \emph{QUADRUPED} ROBOT DENGAN INTERAKSI BERBASIS \emph{TASK LEVEL}}
\end{center}
\addcontentsline{toc}{chapter}{ABSTRAK}
% Menyembunyikan nomor halaman
\thispagestyle{empty}

\begin{flushleft}
  \setlength{\tabcolsep}{0pt}
  \bfseries
  \begin{tabular}{ll@{\hspace{6pt}}l}
  Nama Mahasiswa / NRP&:& Mochammad Hilmi Rusydiansyah / 5024211008\\
  Departemen&:& Teknik Komputer FTEIC - ITS\\
  Dosen Pembimbing&:& 1. Muhtadin, S.T., M.T.\\
  & & 2.  Prof. Dr. Ir. Mauridhi Hery Purnomo, M.Eng.\\
  \end{tabular}
  \vspace{4ex}
\end{flushleft}
\textbf{Abstrak}

% Isi Abstrak
Seiring dengan perkembangan teknologi, kebutuhan akan sistem otomatisasi yang fleksibel dan adaptif semakin meningkat.
Dalam industri manufaktur, gudang logistik, dan layanan rumah tangga,
robot digunakan untuk mengurangi ketergantungan pada tenaga manusia dalam tugas-tugas seperti pick-and-place.
Namun, sebagian besar sistem robotik saat ini masih memiliki keterbatasan dalam hal mobilitas dan
kemampuan beradaptasi terhadap lingkungan yang dinamis.
Penelitian ini bertujuan untuk merancang dan mengembangkan sistem robotik yang mengintegrasikan robot lengan dengan
platform robot \emph{quadruped-legged} untuk meningkatkan fleksibilitas dan kemampuan manipulasi objek.
Studi ini mencakup perancangan sistem mekanik dan elektronik, integrasi sensor untuk deteksi objek,
serta pengembangan algoritma kendali yang memungkinkan robot melakukan tugas pick-and-place secara otonom.

\vspace{2ex}
\noindent
\textbf{Kata Kunci: \emph{Pick-and-Place, Quadruped-Legged, Robot Lengan}}