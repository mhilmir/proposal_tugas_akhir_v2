\chapter{HASIL YANG DIHARAPKAN}

\section{Hasil yang Diharapkan dari Penelitian}

Hasil yang diharapkan dari penelitian ini adalah terciptanya sebuah sistem robotik berbasis
\emph{Human-Robot Interaction} (HRI) yang memungkinkan operator (manusia) untuk berperan aktif
dalam proses pemilihan objek yang akan diambil oleh robot lengan yang terpasang pada robot
\emph{quadruped}. Dengan adanya \emph{Graphical User Interface} (GUI), sistem ini akan menampilkan
deteksi objek dalam bentuk bounding box, sehingga operator dapat memilih objek secara langsung melalui
antarmuka yang intuitif, menghilangkan kemungkinan kesalahan robot dalam pengambilan keputusan atau
memilih objek yang tidak diinginkan. Setelah pemilihan objek dilakukan, sistem akan mengoptimalkan
\emph{pose grasping} menggunakan GraspNet untuk memastikan genggaman yang stabil dan akurat. Penelitian
ini juga menargetkan pengembangan algoritma sistem \emph{task-level} yang mampu mengintegrasikan berbagai komponen,
termasuk persepsi visual, pemrosesan data \emph{grasping}, kendali robot lengan, serta navigasi \emph{quadruped},
sehingga robot dapat bekerja secara efisien dalam menjalankan tugasnya.

\section{Hasil Pendahuluan}
Sejauh ini, penulis...