\chapter{HASIL YANG DIHARAPKAN}

\section{Hasil yang Diharapkan dari Penelitian}

Hasil yang diharapkan dari penelitian ini adalah terciptanya sebuah sistem robotik berbasis
\emph{Human-Robot Interaction} (HRI) yang memungkinkan operator (manusia) untuk berperan aktif
dalam proses pemilihan objek yang akan diambil oleh robot lengan yang terpasang pada robot
\emph{quadruped}. Dengan adanya \emph{Graphical User Interface} (GUI), sistem ini akan menampilkan
deteksi objek dalam bentuk bounding box, sehingga operator dapat memilih objek secara langsung melalui
antarmuka yang intuitif, menghilangkan kemungkinan kesalahan robot dalam pengambilan keputusan atau
memilih objek yang tidak diinginkan. Setelah pemilihan objek dilakukan, sistem akan mengoptimalkan
\emph{pose grasping} menggunakan GraspNet untuk memastikan genggaman yang stabil dan akurat. Penelitian
ini juga menargetkan pengembangan algoritma sistem \emph{task-level} yang mampu mengintegrasikan berbagai komponen,
termasuk persepsi visual, pemrosesan data \emph{grasping}, kendali robot lengan, serta navigasi \emph{quadruped},
sehingga robot dapat bekerja secara efisien dalam menjalankan tugasnya.

\section{Hasil Pendahuluan}
Sejauh ini, penulis telah melakukan studi literatur terkait \emph{tools} dan \emph{library} yang digunakan dalam penelitian ini.
\emph{Tools} dan \emph{library} tersebut antara lain GraspNet sebagai \emph{framework} berbasis \emph{deep learning} yang dirancang untuk
melakukan prediksi \emph{pose grasping} bagi lengan robot, ROS sebagai sebuah \emph{framework open-source} yang membantu pengembangan,
pengendalian, serta komunikasi antar komponen sistem robotik, serta QT sebagai \emph{framework} pengembangan aplikasi yang digunakan
untuk membangun antarmuka grafis (GUI). Selain itu, studi literatur juga dilakukan terkait \emph{hardware} yang akan digunakan dalam penelitian,
yaitu kamera Intel Realsense, robot \emph{quadruped} DeepRobotics Lite3, serta robot lengan (Open Manipulator - X).

[TAMPILKAN GAMBAR TOPIC 2 STREAM TSB, dan gambar streamingannya]

Dalam proses implementasi GraspNet, penulis berada dalam tahap konfigurasi kamera yang akan digunakan yaitu Intel Realsense.
Kamera ini merupakan hal dasar yang diperlukan untuk melakukan \emph{grasp detection}. Terdapat tiga output yang dikeluarkan oleh kamera,
yaitu \emph{RGB Stream}, \emph{Depth Stream}, dan Odometri. Dari ketiganya, hanya dua yang digunakan nantinya yaitu \emph{rgb stream}
dan \emph{depth stream}. Penulis telah melakukan instalasi ROS-wrapper berbentuk package realsense2_camera untuk kedua \emph{stream} ini.
Dari gambar x dapat dilihat terdapat dua topic ROS yang merepresentasikan \emph{stream} untuk rgb dan depth. Dengan kedua topic ini,
maka data streamnya dapat diteruskan ke program GraspNet sebagai data persepsi untuk dilakukan \emph{grasp detection}. Sedangkan gambar x
memperlihatkan hasil streaming dari kamera realsense yang ditampilkan dengan rqt tools.

Selain itu, penulis juga telah mengintegrasikan sistem motion planning untuk robot lengan. Sistem motion planning yang digunakan adalah moveit,
sedangkan robot yang digunakan untuk percobaan ini adalah Open Manipulator - X. Sistem motion planning ini merupakan sistem yang membuat
peneliti dapat mengatur bagaimana robot lengan bergerak. Karena menggunakan ROS, robot lengan digerakkan dengan cara mengirimkan informasi
ke beberapa topic, service, dan action yang terkait seperti di bawah ini.

[SS AN TOPIC SERVICE ACTION]

Selain itu, opencr sebagai perantara untuk laptop yang menjalankan ROS dengan robot lengan harus di flash firmware yang berfungsi sebagai
usb to dxl. Dalam sistem ROS, komunikasi dengan motor Dynamixel dilakukan melalui protokol serial (RS-485 atau TTL),
sedangkan sebagian besar komputer hanya memiliki antarmuka USB. USB to DXL berfungsi sebagai jembatan dengan mengonversi data dari
USB ke protokol komunikasi yang digunakan oleh motor Dynamixel. Pada Open Manipulator yang dikendalikan dengan ROS,
USB to DXL memungkinkan komunikasi antara PC atau single-board computer (seperti Raspberry Pi) dengan motor Dynamixel
melalui port USB, menggunakan perangkat lunak seperti dynamixel_workbench atau ros_control untuk mengirim perintah gerakan,
membaca status motor, dan menjalankan algoritma kontrol dalam lingkungan ROS.

Dalam validasi sistem motion control, robot lengan dapat dikontrol dengan dua cara, yaitu menggunakan keyboard dan GUI MoveIT.
Dalam kontrol keybord, robot dikendalikan dengan mengontrol tiap servo. Terdapat lima servo.
Dalam kontrol GUI, robot dapat melakukan motion planning sesuai dengan 

motion planning ini merupakan salah satu tahap dalam grasping,
graspnet tidak include motion planning, hanya sampai mendapat 6-dof info