\chapter{TINJAUAN PUSTAKA}

% Ubah konten-konten berikut sesuai dengan isi dari tinjauan pustaka
\section{Hasil penelitian/perancangan terdahulu}

Pada penelitian yang berjudul "\emph{Task-Level Intelligent Human-Robot Interaction
for Assisting Multi-objective Autonomous Grasping Decision with Quadruped Robots}"
\parencite{QifanZhang_tlihrifamoagdwqr}, Zhang et al. mengusulkan pendekatan
interaksi manusia-robot (HRI) berbasis \emph{task-level} untuk meningkatkan kemampuan robot \emph{quadruped}
dalam mengambil keputusan saat melakukan \emph{grasping} terhadap beberapa objek secara otonom.
Dalam penelitian ini, sistem yang dikembangkan bertujuan untuk mengatasi keterbatasan sistem
\emph{grasping} otonom pada robot \emph{quadruped} yang umumnya memiliki kapabilitas pengambilan keputusan
yang terbatas serta minim interaksi dengan operator manusia. Penelitian ini merancang sebuah terminal kontrol
yang dilengkapi dengan layar sentuh, memungkinkan operator untuk secara intuitif menentukan
pusat pencarian objek melalui tampilan video yang diterima dari robot.
Eksperimen yang dilakukan dalam lingkungan nyata menunjukkan bahwa metode ini efektif dalam
menangani skenario \emph{multi-target} \emph{grasping} dan meningkatkan pengambilan keputusan robot \emph{quadruped}
saat menghadapi berbagai objek.

% Terminal kontrol menerima umpan video langsung dari robot melalui pemancar video, kemudian menampilkan gambar yang telah
% didekodekan dalam antarmuka grafis (GUI). Dengan memilih titik tertentu dalam gambar sebagai pusat
% pencarian, sistem pengenalan target yang diterapkan akan menggunakan prinsip jarak Euclidean terpendek
% untuk mencari objek yang paling relevan dalam area sekitar titik yang dipilih. Setelah
% target grasping ditentukan, sistem kemudian secara otomatis memulai perencanaan lintasan
% serta deteksi grasping untuk mengeksekusi tugas manipulasi objek.  

Dalam penelitian lain yang berjudul "\emph{Scene Prediction and Manipulator Grasp Pose
Estimation Based on YOLO-GraspNet}"\parencite{LiWanyan_spamgpeboyg}, Wanyan et al. mengusulkan
algoritma estimasi posisi berbasis prediksi skenario yang disebut YOLO-GraspNet untuk
meningkatkan akurasi dan kecepatan dalam proses \emph{grasping} oleh lengan robot, terutama dalam
menangani objek dengan bentuk tidak beraturan di lingkungan yang kompleks.
Metode yang dikembangkan terdiri dari dua tahap utama. Pada tahap pertama, model YOLOv5s
digunakan untuk mengidentifikasi dan menentukan lokasi target \emph{grasping}, serta menandai informasi
kedalaman dari area target yang telah terdeteksi. Dengan demikian, hanya area yang relevan
yang akan diproses lebih lanjut, sehingga mengurangi jumlah data yang perlu diproses dan
meningkatkan efisiensi sistem. Selanjutnya, pada tahap kedua, jaringan GraspNet digunakan
untuk memproses data dari area yang telah ditandai guna memperkirakan \emph{pose} \emph{grasping} yang optimal.
Dengan mengombinasikan keunggulan YOLOv5s dalam deteksi objek yang cepat dan akurat serta
kemampuan GraspNet dalam prediksi \emph{pose} \emph{grasping} yang presisi, metode ini dapat mengatasi
tantangan utama yang dihadapi oleh pendekatan konvensional, seperti jumlah data masukan yang besar,
kecepatan perhitungan yang lambat, dan ketidakakuratan dalam menentukan posisi objek dengan bentuk kompleks.

\section{Teori/Konsep Dasar}

% \subsection{Hukum Newton}

% % Contoh penggunaan referensi dari pustaka
% Newton pernah merumuskan \parencite{Newton1687} bahwa \lipsum[8]
% % Contoh penggunaan referensi dari persamaan
% Kemudian menjadi persamaan seperti pada persamaan \ref{eq:FirstLaw}.
% % Contoh pembuatan persamaan
% \begin{equation}
%   % Label referensi dari persamaan yang dibuat
%   \label{eq:FirstLaw}
%   % Baris kode persamaan yang dibuat
%   \sum \mathbf{F} = 0\; \Leftrightarrow\; \frac{\mathrm{d} \mathbf{v} }{\mathrm{d}t} = 0.
% \end{equation}
% \lipsum[9]

\subsection{Robot \emph{Quadruped}}

\subsection{Robot Lengan}

\subsection{Human-Robot Interaction}

\subsection{Grasping Pose Detection}

\subsection{Pengolahan Citra}

\subsection{ROS}

\subsection{GraspNet}