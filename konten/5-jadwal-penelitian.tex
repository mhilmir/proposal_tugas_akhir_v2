\chapter{JADWAL PENELITIAN}

% Ubah tabel berikut sesuai dengan isi dari rencana kerja
\newcommand{\w}{}
\newcommand{\G}{\cellcolor{gray}}
\begin{table}[H]
  \captionof{table}{Tabel timeline}
  \label{tbl:timeline}
  \begin{tabular}{|p{3.5cm}|c|c|c|c|c|c|c|c|c|c|c|c|c|c|c|c|}

    \hline
    \multirow{2}{*}{Kegiatan} & \multicolumn{16}{|c|}{Minggu}                                                                       \\
    \cline{2-17}              &
    1                         & 2                             & 3  & 4  & 5  & 6  & 7  & 8  & 9  & 10 & 11 & 12 & 13 & 14 & 15 & 16 \\
    \hline

    % Gunakan \G untuk mengisi sel dan \w untuk mengosongkan sel
    Studi Literatur           &
    \G                        & \G                            & \w & \w & \w & \w & \w & \w & \w & \w & \w & \w & \w & \w & \w & \w \\
    \hline

    Instalasi Kamera Pada Robot Lengan           &
    \w                        & \w                            & \G & \G & \w & \w & \w & \w & \w & \w & \w & \w & \w & \w & \w & \w \\
    \hline

    Implementasi GraspNet Pada Robot Lengan              &
    \w                        & \w                            & \w & \w & \G & \G & \G & \w & \w & \w & \w & \w & \w & \w & \w & \w \\
    \hline

    Algoritma Pemilihan Objek Grasping       &
    \w                        & \w                            & \w & \w & \w & \w & \G & \G & \w & \w & \w & \w & \w & \w & \w & \w \\
    \hline

    Perancangan Desain Mounting Robot Lengan Pada Quadruped       &
    \w                        & \w                            & \w & \w & \w & \w & \w & \w & \G & \G & \w & \w & \w & \w & \w & \w \\
    \hline

    Algoritma Sistem Task-Level       &
    \w                        & \w                            & \w & \w & \w & \w & \w & \w & \w & \w & \G & \G & \G & \G & \w & \w \\
    \hline

    Testing - Bug Fixes       &
    \w                        & \w                            & \w & \w & \w & \w & \w & \w & \w & \w & \w & \w & \w & \w & \G & \G \\
    \hline
  \end{tabular}
\end{table}

Pada \emph{timeline} yang tertera di Tabel \ref{tbl:timeline}, menggambarkan perkiraan jenis kegiatan
yang akan dilakukan selama 16 minggu perkuliahan. Dengan pembagian waktu yang telah direncanakan secara
sistematis, diharapkan penelitian ini dapat berjalan dengan baik dan selesai tepat waktu. Pada minggu 1
hingga minggu 2, peneliti akan melakukan studi literatur untuk memahami konsep-konsep dasar yang diperlukan
dalam penelitian ini, termasuk terkait \emph{grasping} pada robot lengan, pemrosesan citra untuk deteksi objek,
serta perancangan algoritma \emph{task-level} yang akan digunakan. Studi literatur ini menjadi landasan penting
sebelum masuk ke tahap implementasi teknis. Memasuki minggu 3 dan 4, peneliti akan melakukan instalasi kamera
pada robot lengan, yang mencakup proses pemasangan perangkat keras serta konfigurasi awal agar kamera dapat digunakan
untuk mendeteksi objek. Pada tahap ini juga dilakukan pengujian awal terhadap sistem kamera untuk memastikan data visual
yang dihasilkan cukup akurat untuk tahap selanjutnya.

Pada minggu 5 hingga minggu 7, peneliti akan mulai mengimplementasikan GraspNet pada robot lengan untuk menentukan
\emph{pose} \emph{grasping} yang optimal terhadap objek yang dipilih oleh operator. Pada minggu 7 dan 8, penelitian
berlanjut ke tahap pengembangan algoritma pemilihan objek \emph{grasping}, yang bertujuan untuk memungkinkan operator
memilih objek secara langsung melalui GUI yang menampilkan \emph{bounding box} dari objek-objek yang terdeteksi.
Pada minggu 9 dan 10, peneliti akan melakukan perancangan desain \emph{mounting} robot lengan pada quadruped.
Tahap ini mencakup desain mekanik dan analisis kestabilan agar \emph{mounting} dapat menopang robot lengan
dengan baik tanpa mengganggu mobilitas quadruped. Pada minggu 11 dan 14, peneliti akan mengembangkan algoritma
sistem \emph{task-level} yang akan mengoordinasikan seluruh komponen dalam sistem, termasuk pemrosesan visual,
pemilihan objek, perhitungan \emph{grasping}, dan kendali pergerakan robot lengan serta quadruped. Akhirnya,
pada minggu 15 hingga 16, penelitian akan memasuki tahap \emph{testing} dan \emph{bug fixes} untuk memastikan
bahwa sistem bekerja dengan baik dalam berbagai skenario. Dengan adanya perencanaan ini, diharapkan penelitian
dapat berjalan secara efektif dan menghasilkan sistem yang dapat berkontribusi dalam pengembangan robot berbasis
\emph{Human-Robot Interaction} (HRI) yang lebih adaptif dan efisien.
