\chapter{PENDAHULUAN}

\section{Latar Belakang}

% Ubah paragraf-paragraf berikut sesuai dengan latar belakang dari tugas akhir
Industri saat ini mengalami transformasi besar dengan semakin meluasnya adopsi otomatisasi
dan robotika untuk meningkatkan efisiensi, produktivitas, serta keselamatan kerja\parencite{Miftachul_podrded}.
Berbagai sektor industri seperti manufaktur, logistik, kesehatan, dan pertanian, semakin bergantung pada otomatisasi
untuk memenuhi permintaan pasar yang terus meningkat dan persaingan yang semakin ketat.
Dalam konteks ini, robot memainkan peran penting sebagai mesin yang dirancang untuk
menjalankan tugas secara otomatis, baik secara mandiri maupun dengan intervensi manusia.
Robot didefinisikan sebagai perangkat mekanis yang dapat diprogram untuk melakukan berbagai tugas,
mulai dari perakitan komponen di lini produksi hingga pengelolaan inventaris di gudang.
Dengan kemampuan bekerja tanpa lelah, akurasi tinggi, serta integrasi dengan kecerdasan buatan (AI)
dan sistem IoT, robot tidak hanya meningkatkan efisiensi operasional tetapi juga
membantu perusahaan mengurangi biaya produksi dan meningkatkan keselamatan kerja.

Berbagai sektor seperti manufaktur, energi, konstruksi, dan logistik semakin bergantung pada robot
untuk menjalankan tugas-tugas yang repetitif, berbahaya, atau membutuhkan presisi tinggi\parencite{WenhuaYuan_rotifiraotsocimgvc}.
Robot lengan, misalnya, telah lama digunakan dalam lini produksi untuk perakitan, pengelasan,
dan pemindahan material dengan kecepatan serta akurasi yang sulit dicapai oleh manusia.
Robot lengan sendiri merupakan salah satu jenis robot industri yang dirancang untuk
meniru gerakan tangan manusia dalam melakukan berbagai tugas, baik yang bersifat sederhana
seperti mengambil dan meletakkan objek (\emph{pick-and-place}) maupun yang lebih kompleks
seperti perakitan komponen dengan presisi tinggi. Robot ini umumnya terdiri dari beberapa segmen
yang dihubungkan oleh sendi (\emph{joint}), yang memungkinkan gerakan fleksibel dan presisi dalam berbagai sumbu.
Dengan adanya aktuator dan sensor, robot lengan dapat beroperasi secara otomatis maupun dikendalikan
oleh manusia untuk menyesuaikan dengan tugas yang diberikan. Saat ini, robot lengan banyak digunakan
dalam industri otomotif, elektronik, manufaktur, hingga medis untuk meningkatkan efisiensi,
mengurangi risiko kecelakaan kerja, dan memastikan kualitas produksi yang konsisten.

Selain robot lengan, terdapat juga robot \emph{quadruped} yang mulai mendapatkan perhatian sebagai
solusi yang fleksibel di lingkungan industri yang kompleks.
Robot \emph{quadruped} sendiri merupakan jenis robot berkaki empat yang dirancang untuk meniru
cara berjalan hewan berkaki empat, seperti anjing atau kuda.
Dengan struktur ini, robot \emph{quadruped} memiliki stabilitas yang lebih baik
dibandingkan robot berkaki dua (bipedal) serta fleksibilitas yang lebih tinggi
dibandingkan robot beroda dalam menghadapi berbagai jenis medan\parencite{LeiWu_doassfqrbog}.
Dengan empat kaki yang dapat bergerak secara independen, robot ini mampu menyesuaikan
langkahnya untuk melewati rintangan, menanjak, atau berjalan di permukaan yang berbatu,
berlumpur, maupun berpasir—sesuatu yang sulit dilakukan oleh robot beroda yang bergantung
pada permukaan datar untuk pergerakan optimal. Selain itu, robot \emph{quadruped} dapat
mempertahankan keseimbangannya dengan lebih baik saat menghadapi perubahan ketinggian
atau ketika berjalan di area sempit dan tidak stabil.
Di industri, robot \emph{quadruped} dapat digunakan untuk inspeksi dan pemeliharaan fasilitas,
terutama di area yang sulit diakses oleh manusia atau kendaraan beroda.
Dalam operasi pencarian dan penyelamatan, robot ini dapat menjelajahi medan yang rusak akibat bencana,
seperti reruntuhan bangunan atau area banjir, guna mencari korban dan mengirimkan bantuan.
Sementara itu, dalam eksplorasi lingkungan kompleks, seperti tambang bawah tanah,
fasilitas nuklir, atau planet lain, robot \emph{quadruped} dapat berperan
sebagai alat pengumpul data dan eksplorasi tanpa membahayakan manusia.
% Robot \emph{quadruped} semakin banyak digunakan dalam berbagai aplikasi berkat
% kemajuan teknologi yang mendukung kinerjanya.
% Perkembangan sensor yang lebih canggih dan terjangkau memungkinkan robot ini
% untuk memiliki persepsi lingkungan yang lebih baik, seperti kemampuan mendeteksi rintangan,
% mengenali objek, dan menyesuaikan pergerakannya secara dinamis.
% Selain itu, aktuator yang semakin efisien dan presisi meningkatkan stabilitas
% serta responsivitas gerakan, sehingga robot dapat beradaptasi dengan berbagai medan dan situasi.
% Kemajuan dalam sistem kontrol juga berperan penting dalam memastikan pergerakan yang halus
% dan koordinasi yang optimal antara keempat kakinya.
% Di sisi lain, integrasi kecerdasan buatan (AI) memungkinkan robot \emph{quadruped}
% untuk belajar dari lingkungannya, mengambil keputusan secara mandiri,
% serta menjalankan tugas dengan tingkat otonomi yang lebih tinggi.
% Kombinasi dari faktor-faktor ini menjadikan robot \emph{quadruped} semakin andal
% dan fleksibel untuk digunakan dalam berbagai bidang,
% mulai dari industri hingga eksplorasi di lingkungan yang menantang.

Pada umumnya, robot \emph{quadruped} yang dijual di pasaran hanya berfungsi sebagai platform mobilitas,
seperti untuk keperluan monitoring dan eksplorasi, tanpa dilengkapi dengan mekanisme aktuator
yang memungkinkan manipulasi objek di sekitarnya. Robot-robot ini biasanya hanya dibekali
dengan sistem sensor untuk navigasi dan persepsi lingkungan, tetapi belum memiliki kemampuan
untuk berinteraksi secara fisik dengan objek lain. Jika robot dilengkapi dengan aktuator
seperti lengan robot dan \emph{gripper}, pengendalian menjadi lebih kompleks karena operator harus
mengontrol tidak hanya pergerakan robot itu sendiri, tetapi juga gerakan lengan secara presisi
agar dapat melakukan tugas manipulasi dengan akurat. Kesulitan ini semakin meningkat ketika
pengendalian dilakukan dari jarak jauh, di mana perbedaan sudut pandang
antara operator dan lingkungan robot dapat menyebabkan kesalahan dalam mengestimasi posisi serta
orientasi objek yang hendak dimanipulasi. Selain itu, keterbatasan dalam umpan balik visual
dan kendala latensi dalam komunikasi juga dapat mempengaruhi efektivitas kontrol,
sehingga manipulasi objek menjadi semakin sulit dilakukan. Oleh karena itu, integrasi aktuator
pada robot \emph{quadruped} memerlukan sistem kontrol yang lebih canggih dan otomatis agar
interaksi dengan lingkungan dapat dilakukan secara lebih intuitif, efisien, serta
mengurangi ketergantungan pada kendali manual yang berpotensi menimbulkan kesalahan operasional.

\section{Rumusan Masalah}

% Ubah paragraf berikut sesuai dengan rumusan masalah dari tugas akhir
Beberapa rumusan masalah yang dijadikan acuan dalam penelitian ini antara lain :

\begin{enumerate}
    \item Robot \emph{quadruped} pada umumnya hanya memiliki kemampuan berupa mobilitas
    dan persepsi terhadap lingkungannya. Namun masih sangat sedikit yang dilengkapi
    dengan kemampuan untuk memanipulasi lingkungannya menggunakan \emph{gripper} untuk tugas \emph{grasping}.
    \item Kontrol untuk menggerakkan robot hingga melakukan \emph{grasping} tidak dapat dilaksanakan dalam
    satu control yang sama, karena umumnya menggunakan \emph{motion control} yang berbeda antara kaki dan \emph{gripper}.
    \item Untuk mengendalikan robot hingga melakukan \emph{grasping}, diperlukan kontrol yang panjang secara manual.

\end{enumerate}

\section{Batasan Masalah atau Ruang Lingkup}

Batasan masalah diberikan peneliti agar penelitian tetap terfokus pada tujuan yang seharusnya. Beberapa batasan masalah untuk penelitian ini antara lain :

\begin{enumerate}
    \item GraspNet digunakan sebagai \emph{framework} untuk robot melakukan \emph{grasping}.
    \item Objek yang digunakan memiliki ukuran serta berat yang cukup untuk dapat diangkat oleh robot jenis Open Manipulator-X.
    \item Pengujian dilakukan di lantai 9 tower 2 ITS.
\end{enumerate}

\section{Tujuan}

% Ubah paragraf berikut sesuai dengan tujuan penelitian dari tugas akhir
Tujuan dari penelitian ini adalah :

\begin{enumerate}
    \item Membuat \emph{hardware} untuk mengintegrasikan \emph{gripper} dan lengan robot pada robot \emph{quadruped}.
    \item Membangun interaksi robot dan manusia dalam level yang bertahap (\emph{task level})
    dengan membagi tahapan mulai dari mendekati objek, mendeteksi objek, memilih objek, hingga \emph{autonomous grasping}.
    \item Mengintegrasikan algoritma terkini untuk \emph{grasping}, misalnya graspnet agar
    lengan robot dan \emph{gripper} dapat melakukan \emph{grasping} secara otonom tanpa kontrol manual dari operator.
\end{enumerate}

\section{Manfaat}

% Ubah paragraf berikut sesuai dengan tujuan penelitian dari tugas akhir
Manfaat dari penelitian ini adalah mengembangkan sistem kontrol yang memungkinkan
robot \emph{quadruped} untuk tidak hanya bergerak dan mengenali lingkungan,
tetapi juga melakukan manipulasi objek secara otonom menggunakan \emph{gripper}.
Dengan mengintegrasikan GraspNet sebagai \emph{framework grasping}, robot dapat mendeteksi,
memilih, dan menggenggam objek secara lebih optimal tanpa memerlukan kontrol manual yang panjang.
Penelitian ini juga berkontribusi dalam pengembangan sistem kontrol yang lebih terkoordinasi
antara pergerakan kaki dan \emph{gripper}, sehingga meningkatkan efisiensi dalam tugas-tugas manipulasi.
Selain itu, hasil dari penelitian ini dapat diterapkan dalam berbagai bidang,
seperti industri manufaktur, logistik, dan eksplorasi lingkungan yang berisiko bagi manusia.